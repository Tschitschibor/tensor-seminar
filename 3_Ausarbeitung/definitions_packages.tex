% !TEX TS-program = pdflatex
% !TEX encoding = UTF-8 Unicode

% This is a simple template for a LaTeX document using the "article" class.
% See "book", "report", "letter" for other types of document.



%%% PAGE DIMENSIONS
%\usepackage{geometry} % to change the page dimensions
%\geometry{a4paper} % or letterpaper (US) or a5paper or....
% \geometry{margin=2in} % for example, change the margins to 2 inches all round
% \geometry{landscape} % set up the page for landscape
%   read geometry.pdf for detailed page layout information
%\usepackage[textwidth=14cm,textheight=20cm]{geometry}
\usepackage[a4paper,%
            left=2.5cm,right=2.5cm,top=2.5cm,bottom=2.5cm,%
            footskip=.6cm]{geometry}

\def\changemargin#1#2{\list{}{\rightmargin#2\leftmargin#1}\item[]}
\let\endchangemargin=\endlist


\usepackage{graphicx} % support the \includegraphics command and options
\usepackage[parfill]{parskip} % Activate to begin paragraphs with an empty line rather than an indent


%%% PACKAGES
\usepackage{booktabs} % for much better looking tables
\usepackage{array} % for better arrays (eg matrices) in maths
\usepackage{paralist} % very flexible & customisable lists (eg. enumerate/itemize, etc.)
\usepackage{verbatim} % adds environment for commenting out blocks of text & for better verbatim
\usepackage{subfig} % make it possible to include more than one captioned figure/table in a single float
% These packages are all incorporated in the memoir class to one degree or another...


\usepackage{float}
\usepackage[all]{xy}
\usepackage{booktabs}
\usepackage{makecell}
\usepackage{enumerate}
\usepackage{scrextend}
\usepackage{mathtools}
\usepackage{yfonts}
\usepackage{setspace} 

\title{\LARGE Seminar zum Thema Numerische multilineare Algebra:\\
\LARGE Tensor Spaces and Numerical Tensor Calculus\\
\vspace{5mm} %5mm vertical space
\large Vortrag: Numerische Verfahren zur Bestimmung des Rangs eines Tensors}

\date{Vortrag vom 13.01.2020}

\author{Tibor Gr{\"u}n}


\makeatletter
% we use \prefix@<level> only if it is defined
\renewcommand{\@seccntformat}[1]{%
  \ifcsname prefix@#1\endcsname
    \csname prefix@#1\endcsname
  \else
    \csname the#1\endcsname\quad
  \fi}
% define \prefix@section
\newcommand\prefix@section{Kapitel \thesection: }
\makeatother


%%% For Math
\usepackage{amsmath}
\usepackage{amsfonts}
\usepackage{amsbsy}
\usepackage{amsthm}
\usepackage{amssymb}

\usepackage{mathtools}
\usepackage{commath}
\usepackage[sc,osf]{mathpazo}

\makeatletter
\renewcommand*\env@matrix[1][*\c@MaxMatrixCols c]{%
  \hskip -\arraycolsep
  \let\@ifnextchar\new@ifnextchar
  \array{#1}}
\makeatother

%%% Math theorem styles
\theoremstyle{definition}
\newtheorem{thm}{Satz}[section]
\newtheorem{lemma}[thm]{Lemma}
\newtheorem{definition}[thm]{Definition}
\newtheorem{rmk}[thm]{Bemerkung}
\newtheorem{alg}[thm]{Algorithmus}


%%% Math equation numbering
\numberwithin{equation}{section}

%%% For graphics
\usepackage{tikz}
\usepackage{pgfbaselayers}

\pgfdeclarelayer{background}
\pgfdeclarelayer{foreground}
\pgfsetlayers{background,main,foreground}


%%% For bibliography
\usepackage[utf8]{inputenc}
\usepackage[nottoc]{tocbibind}
